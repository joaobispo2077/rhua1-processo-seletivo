\documentclass[12pt]{article}
\usepackage{graphicx}
\graphicspath{ {./images/} }
%encoding
%--------------------------------------
\usepackage[T1]{fontenc}
\usepackage[utf8]{inputenc}
%--------------------------------------

%Portuguese-specific commands
%--------------------------------------
\usepackage[portuguese]{babel}
%--------------------------------------

%Hyphenation rules
%--------------------------------------
\usepackage{hyphenat}
\hyphenation{mate-mática recu-perar}
%--------------------------------------
\usepackage{enumerate}

\author{João Vitor Silva Bispo}

\title{Etapa 3}
\begin{document}

\maketitle

\section{Entrevista}
Enfim inicia-se a etapa da primeira entrevista, isso, conforme o candidato prossegue no processo seletivo da Conduent. 
Nessa fase, ocorre uma entrevista individual, no escritório da Conduent, entre o candidato e um recrutador da Agile One.

\begin{figure}[h]
	\centering
	\includegraphics[scale=0.5]{interview}
	\caption{Imagem sem direitos autorais e com fins ilustrativos - Entrevista}
	\label{fig:mesh1}
\end{figure}
Enquanto testes de técnicos visam mensurar o hardskill do candidato, entrevistas visam conhecer melhor das softskills do participante do processo seletivo, essas nomenclaturas são amplamente usadas por empresas e possuem até certa complexidade interna, mas afinal do que se trata o hardskill e a softskill que será avaliada durante a terceira etapa do processo de contratação da Conduent.

\subsection{HardSkill}
HardSkill
Às hardskills são os componentes do conjunto técnico de habilidades de um candidato, são habilidades que podem ser aprendidas, como por exemplo, para um desenvolvedor mobile que visa atuar construindo aplicativos multiplataformas com a linguagem de programação JavaScript, seria um hardskill ele possuir todo o conjunto de tecnologias que aquela vaga precisa, assim possuindo conhecimento em React Native (framework JavaScript para a produção de aplicativos multiplataforma em Android e iOS) estaria suprindo a necessidade de hardskill dessa vaga. 

Ademais, aplicando ao cenário de uma vaga de analista de suporte técnico na Conduent, uma das habilidades técnicas que o candidato necessita possuir é a elevada velocidade em digitação de palavras por minuto (habilidade inclusive que foi testada, esse teste foi referido na segunda etapa do processo em que este documento apresenta).

\subsection{SoftSkill}

Às softskills são o núcleo de habilidades comportamentais de um candidato, que estão mais ligadas à inteligência emocional e formas de ação e reação diante de situações adversas (também podem ser aprendidas como às hardskills).

Vale enfatizar que não existe frequência exata e perfeita de inteligência emocional, mas que para cada função operada pelas profissões demandam formas únicas de pontos ligados às softskills, portanto só é possível encontrar a combinação “mais ideal”. Os componentes das softskills mais requeridas no mercado envolvem:

\begin{itemize}

\item Pensamento criativo
 \subitem Se manifesta em soluções inusitadas que se tornam possíveis com os recursos presentes.
\item Trabalho em equipe
	\subitem Capacidade de desempenhar facilidade em trabalhar com outras pessoas.
\item Espírito de time
	\subitem Capacidade de integrar produtividade ao seu workflow e o do seu time.
\item Empatia
	\subitem Conectado a capacidade de compreender emoções.
\item Gerenciamento de tempo.
	\subitem Conectado a alocação altamente eficiente de tarefas.
\item Comunicação
	\subitem Conectado a facilidade de transmitir ideias coesas, claras e coerentes.
\item Liderança
	\subitem Conectado a habilidade de inspirar e coordenar pessoas.
\item Flexibilidade
	\subitem Capacidade de adaptar-se as mudanças.
\item Proatividade
	\subitem Conectado a sempre prezar pelo excelente resultado mesmo que não solicitado.
\item Solução de conflitos
	\subitem Capacidade de mediar discussões e encontrar resoluções.
\item Solução de problemas
	\subitem Capacidade de desenvolver soluções.
\item Resiliência ou Antifragilidade
	\subitem a resiliência é a capacidade do indivíduo lidar com problemas, reinventar-se a mudanças, ultrapassar os obstáculos ou simplesmente resistir à pressão de situações adversas...

\end{itemize}

\subsection{Antifragilidade}

Essa última softskill vem sendo cada vez mais requerida no mercado de trabalho. E o tal do antifrágil, além do oposto de frágil, é algo que melhora quando está diante de uma situação inesperada. Esse conceito foi criado pelo autor libanês Nassim Nicholas Taleb no livro “Antifrágil: coisas que se beneficiam com o caos”.

\subsection{Há SoftSkill correta para a Conduent?} 

Como supracitado, só é possível encontrar o conjunto de SoftSkills “mais ideal” dado um determinado contexto de vaga, nível e funções, como por exemplo, um Pleno Tech Lead em Desenvolvimento Web, precisa obrigatoriamente ter como SoftSkill uma liderança forte, boa comunicação, empatia e dentre muitas outras. 

Já um Analista de Suporte técnico Júnior, essencialmente além de se comunicar bem e ser resiliente, necessita ter um bom solucionamento de problemas, uma boa habilidade de solução de conflitos e para sanar a pouca experiência necessita-se de um ótimo pensamento criativo. 

Em outras palavras, dependendo da vaga, existe sim habilidades que a Conduent busca num candidato, mas mais importante para um candidato que essas habilidades é a capacidade de desenvolver ela, isso independente da perspectiva da empresa sobre.

\subsection{O caminho para desenvolver uma softskill} 

Diferentemente das hardskills que nascem da junção de conhecimento teórico com conhecimento prático, algumas softskills apenas são possíveis de se aprender na prática, isso simplesmente pela diferença de teorizar ter uma boa comunicação que tenha sido desenvolvida utilizando-se artigos, livros e revistas é um fator, mas efetivamente apenas ter uma boa comunicação é outro fator que faz total diferença tanto na vivência quanto no momento contratação. 
Mas realmente deveria? Um profissional com uma enorme capacidade de autodesenvolvimento caso haja um impulso não deveria ter o mesmo valor se não mais que um concorrente que já atingiu seu ápice em tal habilidade? No dia a dia, para mentes de perspectiva limitada que não vislumbram que a variável contexto seja o ponto decisivo podem ver que em todos os casos somente um dos dois é valoroso; mas na entrevista para Analista de Suporte técnico Júnior do processo seletivo da Conduent se busca mais que habilidades estabelecidas, mas também Talentos (profissionais com grande potencial de crescimento).

\section{A Entrevista determinante}

Dado que está esclarecido quais informações normalmente um recrutador busca entender do candidato numa entrevista, também ficará mais palpável o que o recrutador da Agile One busca num Analista de Suporte Técnico Júnior na primeira entrevista.

Primeiramente, é objetivado traçar o perfil do candidato, o recruta busca entender mais tanto do hardskills do candidato quanto das softskills do mesmo enquanto verifica o fit técnico e o fit cultural do candidato, essas são nomenclaturas que norteiam profissionais de RH (Recursos Humanos) e podem denominar etapas do processo seletivo, sendo a primeira para identificar se às habilidades técnicas e comportamentais do candidato batem com o que é requerido para a vaga na qual está destinado e a segunda para descobrir se as visões do candidato estão pareadas com a missão, visão e valores da organização.

Além desse mapeamento do perfil do participante do processo seletivo, o recrutador após fazer alguns poucos testes psicotécnicos (será explanado mais a frente neste trabalho), também valida a existência de objetivos do candidato relacionados a aquela vaga e principalmente checa se dentro do conjunto de softskills do participante está contida a proatividade, sendo determinante possuir algum destaque nessa habilidade.

Além disso, sem deixar ofuscar-se um requisito muito importante da primeira entrevista é que nos momentos finais o recrutador pergunta se o candidato está confortável para comunicar-se em inglês, e caso o candidato responda que sim, segue uma diálogo em inglês em torno de obter se o participante possui ou não uma mentalidade frente à intermediação de conflitos junto do quão bem ele consegue comunicar-se nessa língua. 

\subsection{Importância do perfil psicológico}
Em geral, foi uma sequência de validação do conjunto das habilidades técnicas e comportamentais do envolvido no processo junto da criação de um mapa do perfil psicológico do candidato para o conferimento se está de acordo com a vaga ou não.

Pode-se concluir, que, embora o processo seja bem estruturado, há uma problemática gigante em iniciar o mapa do perfil psicológico do candidato tão tarde, pelo motivo de que encontrar alguém que não tenha simbiose alguma com a função a longo prazo gerará malefícios tanto ao contratante quanto ao contratado pelo fato do contratado tornar-se infeliz e acumular estresse podendo se agravar em sérios problemas e para o contratante simplesmente obtém menor produtividade por não ter encontrado alguém mais compatível com a vaga em questão e vale enfatizar que muitas vezes o candidato está se descobrindo e apenas finalmente estar naquela posição mostrará a pessoa se ela tem simbiose com a mesma ou não.

Junto disso, acaba evidente que softskills como proatividade e forte mentalidade frente a mediação de conflitos são habilidades que no lugar de terem um peso tão grande, poderiam ser facilmente ensinadas com a prática a algum candidato interessado em ter tais diferenciais.

Basicamente, é uma etapa muito bem feita e serve como exemplo de parte dos processos seletivos, mas deveria ter ocorrido anteriormente e não ter ocorrido antes, é uma grande falha da Conduent.


\section{Etapa 4}
Logo após o candidato conhecer o local em que residem os escritórios da Conduent e ter a sua entrevista,se dá prosseguimento ao processo e chega a etapa em que são realizados exames médicos e exames psicotécnicos. 

Nesses testes médicos visa identificar alguma deficiência aparente no candidato e também servir de registro do estado da saúde do mesmo para a empresa. Já o testes psicotécnicos....

\section{Exame Psicotécnico}

\begin{figure}[h]
	\centering
	\includegraphics[scale=0.5]{psychotechnicalPeople}
	\caption{Imagem sem direitos autorais e com fins ilustrativos - Exame psicotécnico}
	\label{fig:mesh1}
\end{figure}

Em todo território brasileiros, o teste ou exame psicotécnico consistem em uma sequência de atividades com foco avaliativo guiada por um psicólogo a fim de aprovar ou reprovar um participante de um concurso ou à concessão de alguma autorização, como por exemplo a da carteira de motorista. Em testes como esse, normalmente são avaliadas habilidades de:


\begin{itemize}
	\item Controle psicológico e emocional.
	\item Estruturação para a tomada de decisões;
	\item Retenção de memórias.
	\item Níveis de atenção.
	\item Indicadores de concentração.
	\item Coordenação motora
	\item Destreza diante de certas situações
\end{itemize}
Uma grande vantagem que todas as partes envolvidas do processo seletivo possuem na realização de exames psicotécnicos é que há boa probabilidade de estar melhor definido qual o comportamento padrão de uma pessoa e também mapear diferenças entre indivíduos e as reações do mesmos em frente a variadas situações que divergem do cotidiano.




\end{document}




