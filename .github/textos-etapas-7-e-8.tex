% Para facilitar a manutenção é sempre melhore criar um arquivo por capitulo, para exemplo isso não é necessário 

%---------------------------------------------------------------------------------------
\chapter{Período de treinamento}
Ao longo da vida, os seres humanos influenciam e são influenciados pelo meio-ambiente constantemente. É chamada de educação toda a influência que uma pessoa recebe em seu ambiente social, depois a assimila e enriquece seu comportamento. Em geral, a educação representa o preparo pra vida, mas existem vários tipos que variam em seu objetivo, prazo e modo (institucionalização). 

O treinamento é a educação, institucionalizada ou não, que prepara a pessoa para uma função, sendo considerado um meio de se obter um desempenho adequado no cargo em que o funcionário exerce. Diferente da formação e do desenvolvimento profissional, possui objetivos imediatos, buscando oferecer elementos essenciais para a execução de certa função e segue uma pré-programação realizada pela própria empresa ou por uma empresa especializada em treinamento.

O conceito de treinamento envolve quatro formas de mudança de comportamento. São elas: transmissão de informações entre os treinandos (geralmente genéricas), desenvolvimento de habilidades relacionadas ao desempenho do cargo ou de possíveis ocupações futuras, desenvolvimento de atitudes que envolvem a relação entre as pessoas no ambiente de trabalho, e a elevação de conceitos e ideias para facilitar a sua aplicação na prática administrativa ou para ampliar o pensamento dos gerentes.

\section{Objetivos}
O treinamento é um processo de assimilação cultural a curto prazo pelo qual as pessoas adquirem conhecimento, habilidades e atitudes (CHA). Mais especificamente, serve para preparar a pessoa para a execução imediata de tarefas, oferecer oportunidades de desenvolvimento pessoal e aprimorar sua atidude com relação a convivência no ambiente de trabalho e torná-la mais receptiva a ordens.

\section{Ciclo}
O treinamento ocorre em quatro etapas, seguindo uma sequência programada que pode ser renovada anualmente ou semestralmente. São elas: levantamento de necessidades de treinamento (LNT), programação de treinamento, implementação do programa e avaliação dos resultados.

\subsection{Levantamento de Necessidades de Treinamento (LNT)}
Esta primeira etapa corresponde ao diagnóstico do que precisa ser feito, identificando as carências de treinamento na organização, em uma unidade organizacional ou numa atividade específica. É realizada por meio da reunião de informações utilizando algumas técnicas como entrevistas, conferências, avaliações e análises de relatórios.
Os principais métodos de levantamento são: observação, avaliação de desempenho, entrevista de saída (quando um funciónario está deixando a empresa), relatórios periódicos sobre a produção ou resultados, entre outros.

\begin{figure}[h]

\centering
\includegraphics[width=1\textwidth]{figuras/reuniao.jpg}
\end{figure}

\subsection{Programação de Treinamento}
Iniciando-se o processo programado, a prescrição dos meios de se lidar com as deficiências é fundamentado sobre os seguintes aspectos analisados na etapa anterior: 

- Qual a necessidade?

- Onde foi localizada?

- Qual a sua causa?

- É parte de uma necessidade maior?

- Como resolver (separadamente ou combinada)?

- É necessária alguma providência inicial?

- É uma necessidade imediata?

- É uma necessidade temporária?

- Quantas pessoas e atividades serão atingidas?

- Qual o custo provável?

- Qual o tempo disponível?

- Quem irá executar?

- Onde será executado?

Adentrando o planejamento, é essencial se definir a abordagem do treinamento, tanto quanto seu objetivo e seus métodos. Além disso, é executada a definição dos recursos necessários, da população alvo, do horário e local e da relação custo benefício do programa.


\begin{figure}[h]

\centering
\includegraphics[width=1\textwidth]{figuras/planejamento.jpg}
\end{figure}


\subsection{Implementação do programa}
Constituindo a terceira etapa do ciclo do treinamento, a execução do programa pressupõe uma relação de instrutor e aprendiz, seguindo todas as etapas elaboradas no programa (2ª etapa) de acordo com as necessidades levantadas (1ª etapa). Os instrutores, que transmitem o conhecimento, podem ocupar qualquer nível hierárquico da empresa, assim como os aprendizes, que absorvem o que foi transmitido e modificam o comportamento de acordo.



\subsection{Avaliação dos resultados}
A última etapa do processo é a avaliação dos resultados finais, que deve considerar dois pontos principais: determinar até que ponto o treinamento realmente supriu as deficiências necessárias e verificar se esses resultados têm efeito no alcance das metas da empresa. Também é necessário determinar se as técnicas aplicadas são eficientes comparadas às outras alternativas consideradas.

\begin{figure}[h]

\centering
\includegraphics[width=1\textwidth]{figuras/prova.jpg}
\end{figure}

\section{Vantagens e limitações}
Toda a execução do processo de treinamento possui uma série de vantagens que justificam sua importância mas também possui limitações que dificultam um pouco a sua prática. 
Entre as vantagens estão:

- Capacita as pessoas

- Identifica os pontos fortes e fracos delas

- Aumenta a produtividade

- Ajuda a criar um clima organizacional saudável

- Reduz desperdícios e retrabalho

E as limitações incluem:

- Mobilização de recursos

- Incerteza quanto à continuidade do programa

- Dificuldade de remanejamento de pessoas

- Dificuldade de recolocação no mercado




\section{Análise do caso}
No caso da Conduent Brasil, o período de treinamento é de 1 mês e é aplicada uma prova final com 60 questões para analisar os resultados. O canditato passa dessa etapa se tiver no mínimo 80 por cento de acertos e é feita uma análise de segunda chance se o índice de acertos for menor. É um período suficiente para a adaptação do candidato e a última etapa é bem executada por meio da prova no final do processo, que possui um critério rigoroso para aprovação.

\begin{figure}[h]

\centering
\includegraphics[width=1\textwidth]{figuras/mama.jpg}
\end{figure}
%---------------------------------------------------------------------------------------
\chapter{Contrato estabelecido}
Apesar do contrato só se estabelecer nesta etapa, a empresa já possui um vínculo de emprego e já oferece remuneração desde a etapa anterior (de treinamento).

Após a conclusão de todas as etapas do processo ja foram selecionados os melhores candidatos para exercer a função em questão. A etapa final do processo seletivo seria o estabelecimento do contrato com início de 3 meses no período de experiência, que seria o primeiro passo para a pessoa ser admitida dentro da cultura empresarial e do ambiente de trabalho. Realizadas todas as etapas anteriores, as pessoas selecionadas já possuem as capacitações necessárias para trabalhar na empresa e recebem a oportunidade de iniciar ou continuar suas carreiras.

Assim se inicia a jornada do trabalhador recém-admitido em um novo local de trabalho, com novas pessoas, novas funções e principalmente novas possibilidades.

\begin{figure}[h]

\centering
\includegraphics[width=1\textwidth]{figuras/mao.jpg}
\end{figure}
%---------------------------------------------------------------------------------------





