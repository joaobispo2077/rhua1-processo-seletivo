\documentclass[12pt]{article}
\usepackage{graphicx}
\graphicspath{ {./images/} }
%encoding
%--------------------------------------
\usepackage[T1]{fontenc}
\usepackage[utf8]{inputenc}
%--------------------------------------

%Portuguese-specific commands
%--------------------------------------
\usepackage[portuguese]{babel}
%--------------------------------------

%Hyphenation rules
%--------------------------------------
\usepackage{hyphenat}
\hyphenation{mate-mática recu-perar}
%--------------------------------------
\usepackage{enumerate}

\author{Vinicius Almeida Soares}

\title{Etapa 5}
\begin{document}

\maketitle

\section{Primeiros passos na empresa}
Ocorre no primeiro contato em que o candidato foi aprovado, desde como a notícia é dada, até a abertura da conta salário.


\subsection{A confirmação da aprovação do participante}
É de suma importância no processo seletivo. Com a grande quantidade de testes e processos seletivos que os recrutadores fazem diariamente, aspectos essenciais, como a empatia, acabam se perdendo. Atos de humanidade, como avisar que o entrevistado não se encaixa na cultura da empresa ou simplesmente não foi aprovado, acabam não existindo, causando ansiedade desnecessária aos envolvidos no processo.

Na Conduent, há apenas o aviso que o participante da dinâmica foi aprovado, deixando a resposta em aberto para todos os outros que aplicaram para a vaga.

Aos aprovados, é enviado um e-mail contendo todas as informações necessárias para o comparecimento no dia da integração, onde serão apresentados os cargos e a possibilidade de ascensão, bem como benefícios e assinatura dos termos de confidencialidade. É também feita uma ligação, para que seja certificado que o aprovado recebeu o comunicado.

Este processo não é perfeito, e possui pontos que poderiam ser melhorados, entre eles:

\begin{itemize}
\item Informar o rendimento do candidato para o mesmo, independente se ele foi aprovado ou não
\item  Melhorar o detalhamento do e-mail, de modo que seja introduzida a cultura da empresa e suas dinâmicas.
\end{itemize}

\begin{figure}[h]
	\centering
	\includegraphics[scale=0.5]{maos.jpg}
	\caption{Imagem sem direitos autorais e com fins ilustrativos - acordo}
	\label{fig:mesh1}
\end{figure}

\newpage

\subsection{O exame admissional}

Ocorre em um centro médico terceirizado, e nele são feitos vários procedimentos. Geralmente inicia-se com uma pequena entrevista, onde o participante é questionado sobre possíveis doenças ou licenças que ele possa ter tido em empregos passados, ou se já trabalhou em condições perigosas sujeito a materiais nocivos. 

Caso haja requisitos especiais na vaga de emprego, exames complementares também podem ser feitos, tais como: 

\begin{itemize}
\item exame de glicemia
\item eletrocardiograma
\item hemograma
\item audiometria

\end{itemize}

 Porém, na maioria dos cargos oferecidos pela Conduent, isso não é necessário.

O médico do trabalho também analisa os batimentos cardíacos e a pressão arterial do paciente, além de um simples exame de vista para analisar problemas graves na visão. Se tudo estiver em um nível aceitável, é emitido o Atestado Médico de Capacidade Funcional.

\newpage
\subsection{Abertura da conta-salário no banco}

Para a abertura da conta, a Conduent faz uma declaração para o Bradesco, deixando o processo bem simples. O contratado só precisa ir ao banco com a declaração e seus documentos. Seria mais efetivo e rápido se a própria empresa cuidasse deste aspecto, entrando em contato com o Bradesco e abrindo todas as contas necessárias, depois do consentimento de seus funcionários.


\begin{figure}[h]
	\centering
	\includegraphics[scale=0.4]{fazendoconta.jpg}
	\caption{Imagem sem direitos autorais e com fins ilustrativos - contrato}
	\label{fig:mesh1}
\end{figure}
\end{document}
