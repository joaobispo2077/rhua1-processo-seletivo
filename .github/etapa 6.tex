\documentclass[12pt]{article}
\usepackage{graphicx}
\graphicspath{ {./images/} }
%encoding
%--------------------------------------
\usepackage[T1]{fontenc}
\usepackage[utf8]{inputenc}
%--------------------------------------

%Portuguese-specific commands
%--------------------------------------
\usepackage[portuguese]{babel}
%--------------------------------------

%Hyphenation rules
%--------------------------------------
\usepackage{hyphenat}
\hyphenation{mate-mática recu-perar}
%--------------------------------------
\usepackage{enumerate}

\author{Vinicius Almeida Soares}

\title{Etapa 6}
\begin{document}

\maketitle

\section{Integração}
É o primeiro contato do novo funcionário com a cultura da empresa. É muito importante que este contato seja positivo, para que o contratado se sinta confortável e confiante em iniciar essa nova jornada do melhor jeito possível, e para que o contratante sinta que fez a escolha correta. 

A integração na Conduent é simpática e organizada, os coordenadores apresentam a empresa e explicam detalhadamente sobre os tópicos abaixo.


\subsection{Cargos e ascensão}
Neste dia, os cargos são apresentados, e é falado sobre a possibilidade de ascensão.

Um exemplo seria o cargo de analista em suporte ao cliente bilingue, que possui 3 variações:

\begin{itemize}
\item Analista em suporte técnico bilingue Junior
\item Analista em suporte técnico bilingue Pleno
\item Analista em suporte técnico bilingue Senior
\end{itemize}

\newpage

O novo funcionário começará como junior, e caso seu desempenho seja positivo e ele se destaque, em algum tempo ele será pleno, e consequentemente, senior. O salário aumenta com o passar das variações, junto com a responsabilidade. Enquanto um analista junior é responsável por problemas mais simples de suporte, o senior já lida com problemas mais avançados e complicados, e muitas vezes precisa contar com a ajuda dos engenheiros da empresa.

Apesar das alcunhas sofisticadas, o avanço de cargos na conduent não significar um grande aumento na folha de pagamento no final do mês. A diferença entre o salário de um analista junior e senior é minima, mesmo que tenham inúmeras responsabilidades a mais.

\begin{figure}[h]
	\centering
	\includegraphics[scale=0.5]{atendimento.jpg}
	\caption{Imagem sem direitos autorais e com fins ilustrativos - local de trabalho telemarketing}
	\label{fig:mesh1}
\end{figure}

\newpage

\subsection{Benefícios}

A Conduent conta com 3 benefícios principais:



\begin{itemize}
\item Plano de saúde:
\subitem é o benefício que a empresa oferece para que o funcionário conte com apoio em caso de doenças e problemas de saúde, geralmente pode ser expandido para membros da família. O plano de saúde que a Conduent oferece é básico, não abrangendo muitos exames e hospitais.

\item Vale-refeição:
\subitem é o auxílio oferecido para a alimentação em restaurantes e praças de alimentação dentro ou fora do ambiente de trabalho, algumas vezes vem junto com o vale alimentação, que é exclusivo para mercados. A Conduent oferece um vale refeição relativamente baixo, muitas vezes menor que o preço de uma refeição da região.

\item Vale-transporte:
\subitem feito para ajudar o funcionário na questão do transporte para a empresa. Algumas instituições oferecem passagens a mais, para uso recreativo como idas a museus e cinemas. Na Conduent são oferecidas exatamente o número de passagens que o funcionário usa para a locomoção Casa-Trabalho e Trabalho-Casa, sem nenhuma sobra.

\end{itemize}

Seria interessante o aumento dos benefícios, pois são essenciais para os funcionários, ainda mais para os que possuem família para sustentar. O valor e o tipo de plano dos benefícios acabam sendo insuficientes na maioria das situações.O plano de saúde oferecido é limitado, e os vales tem valores bem abaixo do confortável. Muitos funcionários reclamam do baixo valor diário para refeições, sendo menor que o preço de um simples lanche no restaurante do prédio da Conduent.

\newpage
\subsection{Assinatura dos termos de confidencialidade}

A Conduent possui um contrato de confidencialidade, nele o funcionário deve concordar em não informar a terceiros nenhum tipo de informação sobre a empresa e as atividades que ela exerce, após o dia de integração e a assinatura do contrato, a fase de treinamento é iniciada.

O contrato é padrão para as empresas do ramo de suporte ao cliente, nenhuma crítica a ser feita.

\begin{figure}[h]
	\centering
	\includegraphics[scale=0.4]{contrato.jpg}
	\caption{Imagem sem direitos autorais e com fins ilustrativos - assinando contrato}
	\label{fig:mesh1}
\end{figure}
\end{document}
