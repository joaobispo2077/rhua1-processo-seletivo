\documentclass[12pt]{article}
%encoding
%--------------------------------------
\usepackage[T1]{fontenc}
\usepackage[utf8]{inputenc}
%--------------------------------------

%Portuguese-specific commands
%--------------------------------------
\usepackage[portuguese]{babel}
%--------------------------------------

%Hyphenation rules
%--------------------------------------
\usepackage{hyphenat}
\hyphenation{mate-mática recu-perar}
%--------------------------------------
\usepackage{enumerate}

\usepackage{hyperref}
\hypersetup{
    colorlinks=true,
    linkcolor=blue,
    urlcolor=blue,
}

\urlstyle{same}

\begin{document}

\section{Introdução}

O processo seletivo deixou de constituir apenas um meio de contratação de mão de obra em que o maior beneficiado final é o contratante. Na sociedade contemporânea, difunde-se de forma exponencial a cultura de organizações voltadas à valorização do bem-estar do empregado desde a etapa de seleção desse, em que seus aspectos psicológicos e sociaisrecebem uma maior atenção e cuidado por parte dos empregadores. Assim, observa-se que hoje há uma tendência que as organizações valorizem muito mais o indivíduo e suas necessidades, e não apenas suas qualificações técnicas e acadêmicas.

Diante de tal realidade, esse trabalho realiza análise crítica de um processo seletivo que ocorreu no início de 2019 à empresa \emph{Conduent} do Brasil para o cargo de Analista de Suporte Técnico Bilíngue Júnior, no qual o objetivo era prestar suporte de software via telefone aos produtos de uma das maiores empresas de dispositivos \emph{mobile} e \emph{desktop} do mundo.

Essa análise crítica trata, desde o primeiro contato da empresa via plataforma de currículos online, até a efetivação da contratação. Por isso, percorre o total de oito etapas que constituíram o processo seletivo realizando argumentação acerca da validade e/ou efetividade da etapa em questão através da análise do contexto do candidato, da empresa e de contextos mais globais - que podem abarcar questões que tangem o mercado de trabalho, um certo nicho da sociedade, etc.

Dessa maneira, tem-se como foco verificar se a maneira pela qual o processo seletivo foi conduzido proporcionou aos candidatos uma experiência que valorizasse sua existência enquanto seres humanos e profissionais e assim contribuísse para que cada indivíduo atinja seus objetivos de autorrealização. 

Portanto, diante esse viés crítico, fatos, teorias e metodologias foram correlacionados a fim de dar embasamento a cada posicionamento.

 

\end{document}