\documentclass[12pt]{article}
%encoding
%--------------------------------------
\usepackage[T1]{fontenc}
\usepackage[utf8]{inputenc}
%--------------------------------------

%Portuguese-specific commands
%--------------------------------------
\usepackage[portuguese]{babel}
%--------------------------------------

%Hyphenation rules
%--------------------------------------
\usepackage{hyphenat}
\hyphenation{mate-mática recu-perar}
%--------------------------------------
\usepackage{enumerate}

\begin{document}

\part  {Início do processo seletivo}


Dos meios de candidatura

A Conduent possui duas principais maneiras pelas quais candidatos podem se candidatar a vagas: via plataformas de cadastro de currículo online - meio em que a empresa também divulga suas vagas, ou, via indicação de funcionários. Trataremos mais especificamente da primeira maneira, pois o processo seletivo sob análise se deu pelo site vagas.com, entretanto, é relevante colocar em perspectiva ambas as possibilidades pois elas já impactam na experiência do candidato. 

O processo online, que é o mais comum dentro da Conduent, ocorre através de diversos portais de emprego, dentre os quais:

\begin{itemize}
\item www.vagas.com.br
\item www.linkedin.com.br
\item www.indeed.com.br
\item  www.glassdoor.com.br
\item  www.catho.com.br
\end{itemize} 

Tal presença em alguns dos maiores sites de emprego do país gera enorme alcance à empresa e, como consequência, mais chances de candidatos às vagas. Esse é um ponto muito positivo da empresa: a facillidade de encontrar vagas associadas a seu nome nos mais diversos meios online.

Além disso, a divulgação de vagas na rede precede que os pré-requisitos e objetivos dentro da posição divulgada sejam muito bem definidos pela empresa, evitando, assim, que haja um desencontro de expectativas do profissional para com a organização e vice-versa. 

E observa-se que a diversidade  presente na internet é alcançada pela empresa, e tal como o Código de Conduta Empresarial da Conduent" afirma: 

\begin{quote}
A diversidade nos torna mais fortes, pois permite que possamos aproveitar completamente uma força de trabalho global rica em experiência, conhecimento e criatividade.
\end{quote}

 fonte:  https://downloads.conduent.com/content/usa/en/document/Code-of-Conduct-BrazilianPortuguese.pdf

Dessa forma, tessa pluraridade alcançada pela Conduent faz com que a divulgação de candidaturas online tenha mais um aspecto positivo. 

Já em relação a candidaturas por indicação, de maneira geral, podemos observar que manisfestam-se dois aspectos dessas: a diversidade alcançada pela empresa é consideravelmente reduzida, contudo, o perfil de profissional buscado tende a ser o mais próximo daquilo que a vaga demanda. 

O motivo disso pode ser constatado ao se considerar que o profissional que faz a indicação já possui algum nível de vivência na empresa, de sua cultura organizacional, dos desafios enfrentados diariamente e resultados esperados ao desempenhar sua função. Assim, ao conseguir transmitir para o futuro indicado de forma mais personalizada os pré-requisitos da vaga, existe aí uma tendência de que ele esteja mais consciente do que se esperar.  

De certo, é preciso considerar que existe a parcialidade daquele que indica durante a transmissão de informações sobre o ambiente da empresa e do nível de satisfação com o trabalho desempenhado nela. Devido a isso, apesar da tendência citada, pode ocorrer o efeito oposto em alguns candidatos - especialmente aqueles que estão mais emocionalmente ligados ao que lhes indica - e tenham expectativas desniveladas em relação a posição.

Em síntese, temos que a Conduent está muito presente no meio digital, logo, as candidaturas online são as mais comuns, e essas permitem que a empresa alcance um maior e mais diverso número de indivíduos, bem como divulgue de forma objetiva os pré-requisitos e objetivos da função. Em relação as candidaturas por indicação, nota-se que existe uma maior chance de encontrar candidatos mais aptos às funções, porém, com a ressalva de que a percepção daquele que já integra o quadro de funcionários passa a ter papel de grande influência.  
 
Subtópico 2: Canditadura online analisada - realizada através do site www.vagas.com.br
 
No site da empresa www.vagas.com.br a Conduent disponibilizou a vaga de Suporte Técnico Júnior Bilíngue no primeiro semestre de 2019 com o foco em candidatos interessados em trabalhar com suporte técnico de software via telefone para uma grande empresa do ramo de dispositivos mobile e desktop. Por questões de compliance, o nome da empresa para a qual o serviço seria prestado não poderia ser divulgado. Contudo, alguma das principais informações disponibilizadas eram: 
 
 \begin{itemize}
\item Quanto ao objetivo:
	\begin{enumerate}
	\item Operar com atendimento ao cliente para resolver problemas de software em dispositivos mobile 		visando proporcionar a melhor experiência ao cliente.
	\end{enumerate}
\end{itemize}

 \begin{itemize}
\item Quanto aos pré-requisitos:
	\begin{enumerate}
   	\item Possuir ensino médio completo;
   	\item Ter facilidade com tecnologia; 
   	\item Ser criativo e proativo na solução de problemas; 
   	\item Boa desenvoltura para lidar com situações de conflito. 
	\end{enumerate}
\end{itemize}












\end{document}
