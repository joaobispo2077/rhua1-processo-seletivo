\documentclass[12pt]{article}
%encoding
%--------------------------------------
\usepackage[T1]{fontenc}
\usepackage[utf8]{inputenc}
%--------------------------------------

%Portuguese-specific commands
%--------------------------------------
\usepackage[portuguese]{babel}
%--------------------------------------

%Hyphenation rules
%--------------------------------------
\usepackage{hyphenat}
\hyphenation{mate-mática recu-perar}
%--------------------------------------
\usepackage{enumerate}

\usepackage{hyperref}
\hypersetup{
    colorlinks=true,
    linkcolor=blue,
    urlcolor=blue,
}

\urlstyle{same}


\title{Segunda etapa do processo seletivo}
\author{Viviane Queiroz}
\date{\today}


\begin{document}

\maketitle


\section*{Confirmação da aprovação na primeira etapa e próximos passos}

Após cerca de uma semana da realização dos testes online via plataforma do vagas.com, a empresa \emph {AgileOne} entrou em contato via e-mail para informar sobre a aprovação na primeira etapa e dar direcionamento acerca dos próximos passos. 

Houve a comunicação de que seria necessário comparecimento ao escritório da empresa (\emph {AgileOne}) para realização de mais mais testes e houvesse uma apresentação mais profunda da vaga.

Foram realizados novos testes de inglês e português, também de múltipla escolha, mas com o diferencial de serem feitos em papel. Adicionalmente, houve um teste de digitação com duração de três minutos, em inglês. 

Novamente, ambos os testes em papel obtiveram nível de dificuldade considerada baixa frente aos pré-requisitos da vaga, e o ponto de que esse fator poderia dar uma percepção não muito bem acurada aos candidatos do real nível de inglês necessário se manteve. 

No teste de digitação, não foi específicado o nível de velocidade ou palavras por minuto/segundo ideal, entretanto, essa foi uma informação específica coletada pela recrutadora. Logo, mais um quesito que poderia culminar em desclassificação. 

Observamos que esse fator avaliativo pode ser relacionado com o pré-requisito facilidade com tecnologia e sua existência foi um dos pontos positivos da segunda etapa, pois é condizente com uma das habilidades mais utilizadas no dia a dia da função, retratando, assim, bem aos candidatos o que era esperado. 

Já em relação ao detalhamento da vaga e empresa para a qual o serviço seria prestado, foram abordados: 

\begin{enumerate}
   	\item O foco da marca;
	\item A cultura da empresa prestadora do serviço (\emph{Conduent});
   	\item As principais atividades do dia a dia; 
   	\item Possibilidades de ascenção. 
\end{enumerate}

Desde a descrição da vaga, foi reforçado que o \textbf{foco} da marca era proporcionar uma experiência ao cliente do início ao fim do atendimento. Por isso, também foi muito reforçado que o tipo de atendimento não seria aquele mais padronizado comumente observado no setor de \emph {telemarketing}, por exemplo. Essa especificação foi um dos pontos mais interessantes durante a segunda fase, pois foi possível enxergar, em outras palavras, que haveria uma valorização das individualidades de cada pessoa. 

Já em relação à \textbf{cultura organizacional} da \emph{Conduent}, houve enfâse em que e empresa visava criar um ambiente descontraído e com ampla troca de experiências entre os empregados. E dado que a vaga não demandava experiência, esse foi um outro ponto interessante ao deixar os inexperientes mais confortáveis.

E quanto as \textbf{principais atividades} da função, não houve um grande aprofundamento em ferramentas que seriam utilizadas devido as questões de \emph{compliance}, contudo, foi informado a leitura em inglês seria constantemente utilizada e a fala no nível avançado seria obrigatória para que se obtivesse ascenção de cargo. Essa abordagem pareceu se contrapor ao nível das provas apresentada, e mais uma vez, foi possível notar uma falha qualitativa do nível avaliado dos candidatos, o que poderia culminar em certa frustração, conforme já abordado na seção anterior. 

Por fim, ao tratar das \textbf{possibilidades de promoção}, foram detalhados apenas os níveis hierárquicos possível para Analistas de Suporte: 

\begin{enumerate}
   \item Nível Júnior
   \begin{itemize}
     \item Analista de Suporte I
     \item  Analista de Suporte II
   \end{itemize}
   \item Nível Sênior
   \begin{itemize}
     \item Analista de Suporte III
     \end{itemize}
\end{enumerate}

Foi explicado que a principal diferença dentre dos níveis juniores seria um maior número de produtos atendidos enquanto Analista de Suporte II, já na relação comparativa entre júnior e sênior, foi reforçado que haveria maior número de responsabilidades e maior uso do inglês. 

Em ambos os casos houve uma descrição superficial das diferenças entre eles e do que seria propriamente feito por cada um. Nesse ponto, certa ansiedade poderia ser gerada em alguns candidatos pois apesar do ambiente agradável apresentado, havia ainda assim uma nebulosa em relação ao que seria exercido em si. 

Contudo, pode-se entender que, por questões de sigilo, os nomes de ferramentas e procedimentos mais específicos, por exemplo, não poderiam ser mencionados livremente. 

Dessa forma, apesar de um possível desconforto, os candidatos puderem enxergar, na prática, como o \emph{compliance} seria extramemente valorizado pela empresa. O que constituiu, nesse cenário, um outro aspecto positivo dessa etapa do processo seletivo.

SOBRE COMPLIANCE: https://administradores.com.br/artigos/a-importancia-do-compliance-na-gestao-das-organizacoes

\section*{Sobre o perfil dos candidatos presentes na segunda etapa}

Nessa visita ao escritório da \textit{AgileOne}, foi possível constatar que, de fato, havia heteregeonidade entre os candidatos quanto ao sexo, idade, perfil físico e experiência ou não experiência profissional. 

Assim, foi possível constatar que o uso de candidaturas online realmente atinge a diversidade almejada pela empresa e facilitada pela internet. 

Contudo, não havia a presença de Pessoas com Defiência (PCD). A estrutura física da \emph {AgileOne} se dá num edifício comercial da Avenida Paulista - um dos maiores centros comerciais e empresarias do mundo  - com amplas entradas e elevadores, mas o escritório em si possuia salas que não comportariam pessoas com dificuldade de locomoção, por exemplo. 

E apesar de a vaga não ter deixado explicíto que aceitaria a inclusão de PCDs, observar esse fato faz emergir a reflexão sobre a lentidão da  inclusão dessas pessoas em diversos setores produtivos no  Brasil.

Apesar da  lei nº 8.213, de 1991 -  a Lei de Cotas, instituída no Brasil em 1999, que preconiza  um percentual mínimo de PCDs contratados pelas empresas,  o que ocorreu nessa entrevista infelizmente ainda é muito comum do país.

FONTE PCD: \url {https://kenoby.com/blog/contratacao-de-pcd/}

\end{document}